% vim:ft=tex:
%
\documentclass{beamer}
\usepackage{graphicx}
\usepackage{listings}
\usepackage{verbatim}

\let\oldfootnotesize\footnotesize
%\renewcommand*{\footnotesize}{\oldfootnotesize\tiny}

\lstset{
	basicstyle=\oldfootnotesize\ttfamily,
	escapeinside={\%*}{*)}
}

\setbeamertemplate{footline}[]{}
\setbeamertemplate{navigation symbols}{}


\title{
	Denial of Service Attacks
}
\author{
	Andrew DeMaria (muff1nman) --- \texttt{ademaria@mines.edu}
}

\begin{document}
\maketitle
\begin{frame}{Introduction}
	\begin{columns}
		\begin{column}{0.3\textwidth}
			\includegraphics[width=1\textwidth]{images/andrew.JPG}
		\end{column}
		\begin{column}{0.7\textwidth}
			Andrew DeMaria is a senior at Colorado School of Mines in Golden, Colorado
			studying Computer Science. He spent 3 years teaching Cub Scouts at Peaceful
			Valley Ranch as a lifeguard. Afterwards, he was an intern for Lockheed Martin
			under their classified Military Support Program (MSP) where he worked on
			implementing a full Kerberos and LDAP based Authentication and Authorization
			system. There began his interest in information security and he has continued
			diving into the topic since.

			In his spare time, Andrew enjoys the Colorado Wilderness with backpacking,
			climbing, mountaineering, skiing and photography. When the weather does not
			cooperate he likes to hack on a Raspberry Pi.
		\end{column}
	\end{columns}
\end{frame}

\begin{frame}[allowframebreaks]{Agenda}
	\begin{itemize}
		\item Background
			\begin{itemize}
				\item OSI model
				\item TCP/IP
				\item DNS
			\end{itemize}
		\item General Overview
			\begin{itemize}
				\item Purpose
				\item The people behind these attacks
				\item Events in the news
				\item Short History ? % TODO
			\end{itemize}
		\item Tools used?
			\begin{itemize}
				\item Wireshark
				\item scapy
				\item dig
				\item Other built in Linux tools %TODO remove
			\end{itemize}
		\item Layer 2 Attacks
			\begin{itemize}
				\item :( %TODO
			\end{itemize}
			%TODO check layers
		\item Layer 3 Attacks 
			\begin{itemize}
				\item{SYN Food}
				\item{DNS Reflection}
			\end{itemize}
		\item Application Layer Attacks
			\begin{itemize}
				\item :( % TODO
			\end{itemize}
		\item Prevention
	\end{itemize}
\end{frame}

\begin{frame}{SYN Flood}

\end{frame}

\begin{frame}
	\frametitle{SYN Flood... Background}
	\begin{itemize}
		\item TCP, or Transmission Control Protocol is a Layer 2 protocol underlying
			most internet communications today.  It uses a 3 step handshake to
			establish a connection
			\begin{enumerate}
				\item Client will send a $SYN$ to a listening server.  The sequence
					number($seq\_num$) is set to a random number.
				\item Server responds with $SYN$-$ACK$, with the acknowledgement number
					($ack\_num$) set to 1 plus the $seq\_num$.  It sets a new random value
					for $seq\_num$
				\item Client responds with an $ACK$.
					\begin{itemize}
						\item $ack\_num = 1 + seq\_num$
						\item $seq\_num = 1 + ack\_num$
					\end{itemize}
			\end{enumerate}
	\end{itemize}
\end{frame}

\begin{frame}[allowframebreaks]
	\frametitle{SYN Flood... Theory}
	\begin{itemize}
		\item The TCP handshake process requires the server to save
			state before the connection can be established
			\begin{itemize}
				\item Sequence and Acknowledgement numbers need to be saved
				\item State is saved in a Transmission Control Block (TCB)
			\end{itemize}
		\item There is a limited amount of room that the operating system saves for
			TCBs 
	\end{itemize}
	\begin{enumerate}
		\item The attacker sends a $SYN$ packet
		\item The victim sends back a $SYN$-$ACK$
			% The attacker may also spoof the ip address so that his machine does not
			% respond and close the connection
		\item The attacker sends... {\em nothing}
		\item This leaves a half open TCP connection on the server, consuming
			resources
			\begin{itemize}
				\item TCP connections are only flushed out after a period of time, or in
					some implementations, when the TCP pool is depleted
			\end{itemize}
		\item With so many open connections on the server, new legitimate traffic
			cannot get through
	\end{enumerate}
\end{frame}

\begin{frame}
	\frametitle{SYN Flood... Demo}
\end{frame}

\begin{frame}
	\frametitle{SYN Flood... Mitigations}

\end{frame}

\begin{frame}{DNS Reflection Attacks}

\end{frame}

\begin{frame}[fragile,allowframebreaks]
	\frametitle{DNS Reflection Attacks... Background}
	\begin{itemize}
		\item DNS (Domain Name System) provides a mapping of names to IP addresses
			%That way we can keep down on the number of mysql ip addresses we have to
			%memorize :)
			\begin{verbatim}www.google.com -> 74.125.227.112\end{verbatim}
		\item Hierarchal structure with authoritative nodes
			%\begin{verbatim}
									 %.
							%edu        com           ...
				%mines      google   credera
			%\end{verbatim}
		\item Nodes can have several different types of records
			%for which they are authoritative
			\begin{itemize}
				\item \begin{verbatim}A\end{verbatim} % The typical ip record
				\item \begin{verbatim}CNAME\end{verbatim} 
					% For any aliases that also map to the A record.
				\item \begin{verbatim}AAAA\end{verbatim} 
					% ip 6
				\item \begin{verbatim}TXT...\end{verbatim}
			\end{itemize}
		\item Clients / Resolvers interact with DNS servers with Queries
			%Example dig output TODO explain what dig does /show some dig stuff
			\begin{lstlisting}
			;; QUESTION SECTION:
			;www.google.com.			IN	A

			;; ANSWER SECTION:
			www.google.com.		144	IN	A	74.125.227.208
			\end{lstlisting}
		\item Authoritative responses are from the original DNS
		\item Caching may be employed at which point the authoritative bit will not
			be set.
			\begin{itemize}
				\item DNS caches are usually set up by the ISP
					%This reduces a lot of stress so that each person does not have to do a
					%iterative lookup
			\end{itemize}
		\item Queries use UDP %\em{important}
			%Important for the response aspect of the DNS reflection attack... fire and
			%forget
	\end{itemize}
\end{frame}

\begin{frame}
	\frametitle{DNS Reflection Attacks... Theory}
	The behind a DNS reflection attack is to make {\Large large} DNS requests on
	behalf of your victim. 
	\begin{enumerate}
		\item The attacker sends queries, spoofing the victim's IP.  
			\begin{itemize}
				\item Remember.. this is UDP. 
			\end{itemize}
			% As compared to TCP where a handshake would be involved
		\item DNS server responses are sent to the victim
		\item Attack generates so much traffic that the hardware can't even keep up
			%10 Gpbs up is saturated by a 11 Gpbs down
	\end{enumerate}
	However... None of this is possible without open DNS resolvers...
\end{frame}

\begin{frame}[fragile,allowframebreaks]
	\frametitle{DNS Reflection Attacks... Theory... Open Resolvers}
	Open Resolvers are misconfigured DNS servers that will respond to recursive
	requests for domains they are not authoritative for.
	\begin{itemize}
		\item Remember that to be authoritative, the domain must be in that DNS
			server's zone
		\item  It is okay for DNS servers to be recursive.  The caching servers setup by
			your ISP are recursive... but only for requests coming from their zone. 
		\item The problem lies when anyone can make a query to an [open] resolver
	\end{itemize}
	\framebreak
%\end{frame}
%\begin{frame}[fragile]
	%\frametitle{DNS Reflection Attacks... Theory... Open Resolvers}
	% DO IN DEMO?
	An example of a properly configured domain server
	\begin{lstlisting}[breaklines]
	; <<>> DiG 9.8.1-P1 <<>> @216.239.32.10 andrewdemaria.com
	; (1 server found)
	;; global options: +cmd
	;; Got answer:
	;; ->>HEADER<<- opcode: QUERY, status: REFUSED, id: 57548
	;; flags: qr rd; QUERY: 1, ANSWER: 0, AUTHORITY: 0, ADDITIONAL: 0
	;; WARNING: recursion requested but not available

	;; QUESTION SECTION:
	;andrewdemaria.com.		IN	A
	\end{lstlisting}
	\framebreak
%\end{frame}
%\begin{frame}[fragile]
	%\frametitle{DNS Reflection Attacks... Theory... Open Resolvers}
	An example of a open resolver (very misconfigured)
	\begin{lstlisting}[breaklines]
	; <<>> DiG 9.8.1-P1 <<>> @213.128.1.147 google.com +time=3
	; (1 server found)
	;; global options: +cmd
	;; Got answer:
	;; ->>HEADER<<- opcode: QUERY, status: NOERROR, id: 27001
	;; flags: qr rd ra; QUERY: 1, ANSWER: 11, AUTHORITY: 4, ADDITIONAL: 4

	;; QUESTION SECTION:
	;google.com.			IN	A

	;; ANSWER SECTION:
	google.com.		112	IN	A	173.194.113.192
	\end{lstlisting}
	\framebreak
	There are a little over 28 {\em million} open resolvers operating
	today\footnote{ As reported by http://openresolverproject.org/breakdown.cgi on
	2013-08-11}
	\begin{itemize}
		\item It is pretty trivial for attackers to find them
		\item Most businesses / organizations don't even realize that they are
			contributing to the demise of the internet
	\end{itemize}
\end{frame}

\begin{frame}
	\frametitle{DNS Reflection Attacks... Increasing the effectiveness}
	\begin{itemize}
		\item Botnets allow for additional throughput on the attackers size,
			linearly scaling the attack
		\item Extensions to DNS allow for increase response size
			\begin{itemize}
				\item EDNS 0 allows for DNS packets to go beyond the initial 512 byte
					limit up to 4096 bytes
				\item DNSSEC adds fields for keys and signatures for validating DNS
					records
			\end{itemize}
	\end{itemize}
\end{frame}


\begin{frame}
	\frametitle{DNS Reflection Attacks... Demo}
\end{frame}

\begin{frame}
	\frametitle{DNS Reflection Attacks... In the news}
\end{frame}

\begin{frame}
	\frametitle{DNS Reflection Attacks... Mitigation}
	\begin{itemize}
		\item Shutdown all the open resolvers
			\pause
		\item realistically...
			\pause
			Get yourself a good ISP 
			\pause
		\item Third party companies such as Cloudflare specialize in hosting customers
			content and thwarting DNS Reflection attacks
	\end{itemize}
\end{frame}

\end{document}

